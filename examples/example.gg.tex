\documentclass{article}
\usepackage[utf8]{inputenc}
\usepackage{amsmath}

\begin{document}

This an example of math code that will be translated to Geogebra math
code.

\[
\boxed{
  \textbf{Completing the square:}\\
  \\
  \begin{aligned}
    @a@ x^2 + @b@ x
    &= @a@ \left( x^2 + @FractionText(b/a)@ x \right) \\
    &= @a@ \left( x^2 + @FractionText(b/a)@ x +
      @FractionText(b^2/(4*a^2))@ -  @FractionText(b^2/(4*a^2))@ \right) \\
    &= @a@ \left( x + @FractionText(b/(2a))@ \right)^2 
      - @FractionText(b^2/(4a))@
  \end{aligned}
}
\]

\noindent\hrulefill
\begin{verbatim}
FormulaText(Simplify(
"\boxed{
  \textbf{Completing the square:}\\
  \\
  \begin{aligned}
    " + a + " x^2 + " + b + " x
    &= " + a + " \left( x^2 + " + FractionText(b/a) + " x \right) \\
    &= " + a + " \left( x^2 + " + FractionText(b/a) + " x +
      " + FractionText(b^2/(4*a^2)) + " -  " + FractionText(b^2/(4*a^2)) + " \right) \\
    &= " + a + " \left( x + " + FractionText(b/(2a)) + " \right)^2 
      - " + FractionText(b^2/(4a)) + "
  \end{aligned}
}"
))
\end{verbatim}
\hrulefill\\

Every thing outside of display math mode will be ignored. Even inline
math, like $x^2 + b$.


\end{document}
